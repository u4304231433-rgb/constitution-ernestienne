\documentclass{constitution}

\usepackage{fontspec}

\newcommand{\ern}[1]{{\small\fontspec{ErnestFont}#1}}
\newcommand{\erntrad}[1]{\ern{#1} (#1)}

\begin{document}
\maketitle
\tableofcontents
\newpage
Le peuple ernestien proclame solennellement son attachement à la liberté, à la diversité et au droit des peuples à l’autodétermination. En vertu de ces principes, il fonde son idéal sur la devise “différent donc normal” (“\ern{roizôg êk bizârg}”)
\Article{Nature du régime}
\article{}
La Démocratie Ernestienne Normalienne et Supercool (DENS) est un régime démocratique basé à l’École Normale Supérieure de Paris, mais est ouverte à tous. Elle assure l’égalité devant la loi de tous les citoyens sans distinction. 
\article{}
La nationalité est accordée à tous les membres de l’Ernestie dès leur entrée sur le territoire national.
\titre{Principes fondamentaux}
\Article{Symboles}
Les langues officielles de l’Ernestie sont l’ernestien et le français.
L’emblème national est le drapeau ernestique : un ernest violet sur fond jaune.
La devise de l’ernestie est “Différent donc normal”.
Son principe est “par et pour”.
Le sport officiel de l’ernestie est le lancer de \erntrad{chrûk}.
\Article{Citoyenneté}
\article{}
\subarticle{}
Sont citoyens tous les membres de l’Ernestie qui en font la demande écrite (ou prennent le rôle “Citoyen” lors de leur admission sur serveur Discord de l’Ernestie) et qui ont au minimum un jour d’ancienneté dans la DENS, tant qu’il n’ont pas été déchu de leur citoyenneté. 
\subarticle{}
L’ancienneté est la durée écoulée depuis que l’individu a rejoint le serveur  Discord de l’Ernestie.
\articleref{}{Ancienneté}
Les anciens citoyens sont ceux ayant au minimum 5 jours d’ancienneté, 3 participations à des votes, ainsi que quatre jours de résidence lors du mois précédent.
\article{}
\subarticle{}
Est compté comme jour de résidence tout jour où le citoyen effectue une action sur le territoire national, hors de la capitale.
\subarticle{}
Est défini comme territoire national le serveur Discord de l’Ernestie, et le bassin aux Ernests de la Courô du 45 rue d’Ulm, Paris (75005), qui est choisi pour capitale territoriale de l’Ernestie.
\article{}
Lorsqu’aucun citoyen ne vérifie les conditions de l’\reftarget{Ancienneté}, il n’est plus nécessaire d'avoir participé à 3 votes ni d’avoir 5 jours d’ancienneté pour être ancien citoyen.
\Article{Citoyens permanents et honoraires}
\article{}
La présente constitution reconnaît le statut de citoyen permanent au propriétaire du serveur Discord “Ernestie”.
\article{}
\subarticle{}
L’ensemble des citoyens permanents forme le conseil des citoyens permanents ou conseil permanent.
\subarticle{}
Le conseil permanent assure l’exécution des lois. Il est également garant du fonctionnement régulier des pouvoirs publics, de l’indépendance nationale, de l’intégrité territoriale et du respect des traités.
\article{}
Le statut de citoyen permanent peut être acquis par tout ancien citoyen étudiant actuellement à l’Ecole Normale Supérieure de Paris en récompense de son mérite au regard de la DENS. Un ou plusieurs anciens citoyens peuvent être nommés citoyens permanents, sur demande au conseil permanent et suite au vote favorable d’au moins $\frac{2}{3}$ de ses membres.
\article{}
Un citoyen permanent peut être promu citoyen honoraire par un vote à majorité du conseil constitutionnel après avoir siégé au moins un an au conseil des citoyens permanents. Il peut alors conserver son statut de citoyen permanent tant qu’il reste étudiant de l’ENS Ulm. Après cela, ce statut devient purement honorifique et prend le nom de citoyenneté honoraire² (lire “honoraire carré”). Un membre permanent quittant l’ENS devient donc honoraire². Il ne peut y avoir plus de 13 membres honoraires.
\article{}
Un nombre maximal de 5 citoyens permanents (hors honoraires) au conseil permanent est institué.
\article{}
Toute modification du conseil permanent ou changement de statut (vers honoraire) d’un de ses membres doit s’accompagner d’une promulgation au journal officiel.
\article{}
Dans le cas où il n’y a plus de citoyen permanent (hors honoraires²), les membres élus du conseil constitutionnel ont deux jours pour désigner un citoyen permanent parmi eux. S’ils ne parviennent à se mettre d’accord, un citoyen permanent est tiré au sort parmi le conseil élu. S’il n’existe aucun conseiller constitutionnel, un citoyen permanent est tiré au sort parmi les anciens citoyens.
\Article{Constitutionnalité}
\articleref{}{Constitutionnalité}
Toute décision officielle, loi, procédure réglementaire ou officielle, proposition de loi ou de révision constitutionnelle, doit se conformer à la présente constitution.
\articleref{}{Conseil constitutionnel}
\subarticle{}
Un conseil (ou conseil constitutionnel) est établi pour veiller au respect de l’\reftarget{Constitutionnalité}. Ses membres sont tenus à la neutralité et l’impartialité.
\subarticle{}
Le conseil doit annuler toute décision extérieure au conseil, en particulier le vote d’une loi, si 2/3 de ses membres la jugent anticonstitutionnelle.
\subarticle{}
Les votes du conseil s’effectuent pendant une semaine sur le canal approprié. Il est clos si une majorité d’adoption large aux $\frac{2}{3}$ ou de rejet strict au $\frac{1}{3}$ est atteinte.
\article{}
\subarticle{}
Le conseil constitutionnel est composé du conseil permanent et de cinq membres élus. Si la DENS ne compte pas assez de représentants aux postes, ceux-ci seront déclarés temporairement vacants et la majorité absolue sera fondée sur le nombre de postes occupés.
\subarticle{}
Un membre du conseil constitutionnel ne peut y siéger deux fois simultanément. En particulier, un membre ne peut être simultanément élu et permanent.
\subarticle{}
La loi peut réviser le nombre de membres élus d’au maximum cinq postes à la hausse et de deux à la baisse.
De telles révisions doivent être espacées d’au moins un an.
La révision ne peut fixer un nombre de membres élus excédant le nombre de citoyens permanents hors honoraires² au moment du dépôt de la demande de révision. 
Cette révision ne peut se faire moins d’un an après adoption de cette constitution, ou après toute modification de cet article portant sur le nombre de membres élus.
\articleref{Election des membres}{Election}
\subarticle{}
Des candidats individuels citoyens peuvent, à tout moment, se présenter au poste de membre élu au conseil constitutionnel. Ils ne pourront briguer une élection trimestrielle que s’ils se signalent au minimum une semaine avant cette élection.
\subarticle{}
Le conseil élu est à renouvellement intégral trimestriel. Chaque citoyen élu brigue donc un mandat d’au maximum trois mois. Une élection de renouvellement trimestriel se tient les premiers mercredis des mois de Janvier, Avril, Juillet et Octobre pour le premier tour et samedi pour le second tour.
\subarticle{}
\subsubarticle{}
Si, entre deux élections trimestrielles, un poste n’est pas pourvu, une élection partielle est tenue. Le mandat du conseiller élu s’achève cependant à la prochaine élection trimestrielle.
\subsubarticle{}
Si la vacance d’un ou plusieurs postes est constatée par le conseil constitutionnel, celui-ci initie un appel à candidatures pour le pourvoir. Un délai d’une semaine après le premier dépôt de candidature est imposée pour permettre aux autres citoyens de soumettre la leur (ils ne sont alors pas tenus de respecter le délai minimal d’une semaine avant élection pour la soumission de leur candidature). Une fois cette semaine écoulée, le conseil permanent initie le vote dans le canal approprié sous les mêmes modalités que pour un renouvellement trimestriel, en prenant en compte de toutes les candidatures au poste déposées.
\subarticle{}
A chaque renouvellement (trimestriel comme exceptionnel), le scrutin s’opère en deux tours. Le conseil permanent initie le vote du premier tour un mercredi à midi (UTC+1), sur le canal approprié. Ce vote sera clos le vendredi de la même semaine, à midi (UTC+1). Les citoyens peuvent choisir autant de candidats qu’il y a de postes à pourvoir. A l’issue du premier tour, s’il y a $n$ sièges à pourvoir, les $ n+2 $ personnes, au maximum, ayant obtenu le plus de voix et qui souhaitent y participer, peuvent prendre part au second tour.
Si il y a moins de $ n + 2 $ candidats au premier tour, on n’effectue qu’un seul tour. 
De plus, il faut obtenir au minimum 3 voix au dernier tour pour être élu.
\subarticle{Absentéisme}
Un membre du conseil constitutionnel doit le quitter s’il perd la citoyenneté, s’il démissionne, ou s’il n’a participé à aucun vote du conseil constitutionnel lors des trois dernières semaines où de tels votes ont eu lieu. La dernière condition ne s’applique pas aux honoraires². 
Le conseil constitutionnel se doit de mettre en garde ses membres faisant preuve d’absentéisme, et notamment d’effectuer au moins un rappel à l’ordre, par tous les moyens possibles, trois jours avant la fin d'un vote du conseil, entrainant le constat de leur absentéisme effectif.
\subarticle{}
Lorsqu'un (ou plusieurs) membre élu quitte le conseil constitutionnel, une élection a lieu pour le (ou les) remplacer. Le système est celui décrit en \reftarget{Election}.
\articleref{}{QPC}
\subarticle{}
Tout citoyen peut saisir le conseil constitutionnel en formulant une question prioritaire de constitutionnalité (QPC). Il doit, en pratique, déposer sur le canal approprié une demande claire de vérification de la constitutionnalité d’une loi déjà adoptée. Le cas échéant, la loi est suspendue jusqu’à confirmation par le conseil constitutionnel de sa constitutionnalité. Le conseil doit alors agir selon les dispositions de l'\reftarget{Conseil constitutionnel}.
\subarticle{}
Le conseil se réserve le droit de censurer totalement ou partiellement une QPC abusive déposée. 
Le caractère abusif d’une telle QPC doit être être rigoureusement établi par un membre du conseil constitutionnel. La censure de l’examen de la QPC est alors actée. En cas de désaccord avec cette censure, signalé par un autre membre du conseil avant une semaine après cette censure, la QPC déposée est examinée dans ses termes initiaux. 
\article{Déchéance de nationalité}
Si un citoyen ou national représente une menace pour la DENS, expressément constatée par le conseil, celui-ci peut initier un processus de déchéance de nationalité. Le constat s’effectue par un vote express selon les modalités de l’\reftarget{Conseil constitutionnel}, avec un délai de prise de décision raccourci de cinq à deux jours. En cas de consensus du conseil en faveur de la déchéance, celle-ci est actée. Dans le cas contraire, si plus des 2/3 du conseil constitutionnel est en faveur de la déchéance de nationalité, le conseil organise un vote selon les dispositions de l’\reftarget{Propositions de loi}article 9.1, avec un délai de trois jours entre l’émission du vote et sa clôture. Une fois la déchéance de nationalité actée, celle-ci prend immédiatement effet. Elle s’accompagne du bannissement du territoire national.
\Article{Etat d'urgence}
\article{}
Un état d’urgence peut être décrété, en cas de situation exceptionnelle, par un vote des anciens citoyens. Celui-ci est instauré aussitôt la majorité absolue des anciens citoyens atteinte (et pas seulement la majorité absolue des votants). Le vote est annulé s’il n’a pas abouti au bout d’une semaine.
\article{}
Les situations exceptionnelles reconnues sont les suivantes : 
\begin{itemize}
    \item Invasion du territoire national par des forces visant à la destruction de l’intégrité de la DENS.    
    \item Prise de pouvoir personnelle par un citoyen, groupe de citoyen ou élément extérieur à la DENS.
\end{itemize}
\article{}
\subarticle{}
L’état d’urgence donne au conseil des citoyens permanents la possibilité de promulguer toutes les lois qu’il souhaite, dans le respect de la constitution, ainsi que de déchoir tout citoyen non permanent de sa citoyenneté. Il ne peut modifier la présente constitution.
\subarticle{}
Pendant l’état d’urgence, toute prise de décision du conseil des citoyens permanents s’effectue par vote à majorité aux $\frac{2}{3}$ .
\subarticle{}
La déchéance de citoyenneté d’un citoyen permanent est également possible mais doit s’accompagner du vote positif des $\frac{3}{4}$ du conseil permanent.
\article{}
La durée de l’état d’urgence est de trois jours. Celui-ci peut être étendu de trois jours supplémentaires, de la même façon qu’il a été instauré.
\titre{Corps législatif}
\Article{Domaine d'application des lois}
La loi fixe les règles et principes généraux concernant :
\begin{itemize}
    \item les droits civiques et garanties accordées aux citoyens et nationaux pour l’exercice des libertés publiques : la liberté, le pluralisme et l’indépendance des médias ;
    \item les régimes matrimoniaux, successions et libéralités ;
    \item la détermination des crimes et délits ainsi que les peines qui leur sont applicables ; la procédure pénale ; l'amnistie ; la création de nouveaux ordres de juridiction et le statut des magistrats ;
    \item l'assiette, le taux et les modalités de recouvrement des impositions de toutes natures ; le régime d'émission de la monnaie ;
    \item la création de catégories d'établissements publics ;
    \item les garanties fondamentales accordées aux fonctionnaires civils et de police ;
    \item les nationalisations d'entreprises et les transferts de propriété d'entreprises du secteur public au secteur privé ;
    \item l'enseignement ;
    \item la préservation de l’environnement ;
    \item le régime de la propriété, des droits réels et des obligations civiles et commerciales ;
    \item le droit du travail, du droit syndical et de la sécurité sociale.
\end{itemize}
En cas de conflit avec une loi française explicite s’appliquant aux binationaux, cette dernière prévaut. En particulier, le législateur ne peut s’opposer aux codes et lois françaises.
\Article{Dépôt des propositions de loi}
\articleref{}{Article 8.1}
\subarticle{}
Tout citoyen est libre de proposer une loi, en déposant le contenu du texte de loi (sous forme d’un message pour une loi courte, ou d’un pdf pour une loi longue) en créant un post muni de l’étiquette “Proposition de loi” sur le canal “agora” du serveur de l'Ernestie. Ce texte de loi devra comporter une justification rationnelle précise de son utilité.
\subarticle{}
En cas de non constitutionnalité ou de redondance avec une loi déjà existante ou déjà proposée (dans ce cas l’équivalence doit être démontrée) le conseil constitutionnel peut rejeter le dépôt de la proposition en censurant le message déposé. 
La censure du dépôt d’une loi de fait sous les mêmes modalités que la censure d’une QPC (cf. \reftarget{QPC}).
\article{Débats}
\subarticle{}
La présente constitution instaure une assemblée citoyenne (agora), composée de tous les citoyens, sous la forme d’un canal Discord, afin de débattre des propositions de lois avant de les soumettre à un vote.
\subarticle{}
Une durée minimale de deux jours (48h) de débats sans vote est imposée.
\subarticle{}
Durant cette période, seul l’auteur de la proposition de loi est libre de modifier la loi déposée, pour garantir son adéquation avec la volonté générale.
\subarticle{}
Le conseil constitutionnel peut également, à chaque modification de la proposition de loi, censurer cette proposition selon les modalités de l’\reftarget{Article 8.1}.
\article{}
Le conseil et l’auteur de la proposition de loi closent conjointement la procédure de débats sans vote. Le conseil devra s’assurer de la bonne tenue des débats et que les discussions ont bien permis l’expression des points de vue des citoyens. Une fois les débats menés, le conseil se charge de démarrer le vote, en créant un sondage correspondant dans le canal “votes”.
\Article{Adoption des propositions de lois}
\articleref{}{Propositions de loi}
\subarticle{}
La prise de décision est effectuée par les citoyens à la majorité absolue, par le vote à bulletin public d’une proposition de loi sur le canal "votes" du serveur Discord de l'Ernestie.
Ce vote se manifeste concrètement par un sondage, comprenant la loi rédigée dans ses termes juridiques par son auteur conjointement avec le conseil constitutionnel et un résumé clair de cette proposition faisant office de titre. Ce post comprendra le choix multiple "Oui", "Non", "Vote blanc", permettant au citoyen de se positionner par rapport à l'adoption de ce texte.
Les proportions à atteindre pour l'adoption ou le rejet d'un vote seront basées sur l'ensemble des citoyens votants, qui n'auront pas voté blanc.
\subarticle{}
Seuls les citoyens reconnus comme tels avant le début du vote peuvent y prendre part.
\subarticle{}
Après cinq jours à compter de l’heure d’émission du vote de la proposition de loi (UTC+1) par le conseil constitutionnel, la décision est prise à la majorité relative, sous réserve de l’expression d’au moins un citoyen.
\subarticle{}
Toute situation de majorité absolue avant les 5 jours, respectant les conditions de l’article final conduit à l’adoption immédiate de la proposition de loi.
\article{}
Les lois adoptées sont publiées au journal officiel. La présente constitution crée le canal Discord de ce journal officiel.
\article{}
Les lois ne s’appliquent que 24 h après leur adoption, laissant un délai aux citoyens pour déposer une question prioritaire de constitutionnalité.

\titre{Corps de l'Etat}
\Article{Fonctions administratives et policières}
\article{}
Les responsables administratifs, définis dans les sous articles suivants, sont nommés par le conseil constitutionnel. Le conseil peut révoquer leur mandat à tout moment. Ces responsables peuvent représenter le conseil et doivent lui rendre compte de chacune de leurs interventions officielles. 
\article{}
Le secrétaire d’État est chargé de promulguer les textes de loi adoptés, la version révisée de la constitution après chaque modification ou encore le résultat de chaque désignation d’un citoyen ou élection à un poste officiel au journal officiel. Il peut également organiser des votes, s'il est mandaté par le conseil constitutionnel.
\article{}
Les ambassadeurs servent d’intermédiaires avec les autres nations mais ne peuvent prendre de décisions. Ils doivent rendre compte au conseil de chacune de leurs actions officielles.
\article{}
Le conseil constitutionnel nomme un commissaire général, chargé de veiller à la bonne application de la loi. Les codes procéduraux encadrent l’exercice de l'autorité policière.
\article{}
Les procureurs généraux possèdent le statut de responsables administratifs. Leur rôle est défini dans les codes procéduraux.

\titre{Autorité judiciaire}
\Article{Justice}
Nul ne peut être arbitrairement condamné ou subir une sanction arbitraire.
\Article{Procureurs}
\article{}
Les procureurs généraux sont nommés par le conseil, sélectionnés parmi les citoyens.
\article{}
Le conseil constitutionnel devra veiller à nommer un nombre suffisant de procureurs généraux, afin d’assurer l’impartialité et la pluralité des points de vue dans la représentation judiciaire, et la soutenabilité du système judiciaire.

\titre{Modification et révision constitutionnelle}
\Article{Révision constitutionnelle}
\article{}
Toute modification ou révision constitutionnelle doit s’accompagner d’un vote à majorité au troisième quartile (à plus de ¾ en proportion de voix en faveur de ladite révision) des citoyens et d’une validation du projet de révision par le conseil constitutionnel selon les dispositions de l’\reftarget{Conseil constitutionnel}.
\article{}
Le dépôt d’une demande de révision constitutionnelle doit s’effectuer sur l'agora, accompagné de la spécification précise du point constitutionnel à réviser et du tag "Révision constitutionnelle". L’organisation des débats et du vote est alors identique à celle précédant le vote d’une loi usuelle.

\titre{Relations internationales}
\Article{Traités et accords internationaux}
\article{}
Les ambassadeurs, mandatés par le conseil négocient et ratifient les traités et accords internationaux selon la ligne fixée par le conseil, en accord avec l’opinion publique et l’intérêt général.
\article{}
Les accords et traités internationaux ne peuvent être ratifiés ou approuvés qu’en vertu d’une loi.
Ils ne prennent effet qu’après avoir été ratifiés ou approuvés. 
Nulle cession, nul échange, nulle adjonction de territoire n’est valable sans le consentement des populations et autres espèces animales intéressées.
\article{}
La DENS reconnaît la juridiction de la cour pénale internationale ainsi que le droit international.
\article{}
Les traités approuvés ont autorité supérieure à celle des lois, sous réserve de leur application par l’autre partie.

\titre{L'identité de la DENS}
\Article{Valeurs de la DENS}
La DENS promeut les valeurs d’égalité de tous les citoyens et de respect de tous.
\Article{Pluralité linguistique}
\article{}
Nul ne peut être discriminé pour son usage d’une des deux langues officielles plutôt qu’une autre. 
\article{}
En particulier, les textes officiels doivent, en cas de demande par un citoyen, être traduits dans les deux langues dans un délai d’au plus un mois. Toute proposition de loi doit être publiée dans au moins une des deux langues officielles. De plus, le conseil constitutionnel peut s’opposer à l’adoption d’une loi dont il est manifeste qu’une partie conséquente de la population n’a pas été en mesure d’en comprendre le sens. Cela comprend :
\begin{itemize}
    \item les textes dont la formulation prête volontairement à confusion
    \item les textes en ernestien non traduits en français ou en cas de mensonge sur leur traduction en français
    \item les textes dont les versions dans les différentes langues ne correspondent pas
\end{itemize}
\article{}
Les citoyens, et surtout les officiels bilingues veilleront à la bonne compréhension, par chacun, des textes, décrets et communications officielles de la DENS. Tout usage fallacieux de la langue ernestienne est déclaré anticonstitutionnel.
\end{document}
